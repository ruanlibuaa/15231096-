% !Mode:: "TeX:UTF-8"

\chapter{绪论}

\section{课题研究背景和意义}
作为一门重要的基础学科,数学为其它学科提供了重要的理论支撑,掌握了数学知识就意味着掌握了领悟一种现代科学的理论和工具,学到了一种理性思维模式。我们在受教育的每个阶段几乎都要学习数学,数学不仅为学生进一步学习提供了必要准备,而且在学生终身发展的过程中也有不可替代的作用。尤其是小学数学教学在培养学生逻辑思考能力,开发智力和非智力因素上的作用不容小觑。因此从小培养学生学习数学的兴趣,建立起学好数学的自信心就显得尤为重要。家庭作业作为课堂教学的延伸,既能帮助学生巩固完善课堂所学知识,形成数学能力,又能帮助老师了解学生对知识的掌握情况,合理安排教学计划,作业的分析反馈中含有巨大的教育价值。传统的批改作业方式使老师埋在作业堆里做着机械重复的工作,老师花费大量时间在作业的批改上,却没有实现信息的反馈,对学生的反思进步没有很大的帮助。班级人数众多,每个人存在的问题经常没有记录也无法查看,很难对出错问题进行全面分析,也不能对每个同学的错误进行一一指导,而且长期进行纸面上的数学练习,学生可能会感到枯燥。随着以4G为代表的移动通信技术的迅速升级和智能手机的全面普及,借助移动终端的便利性,短、平、快的学习方式——“碎片化学习”逐渐流行起来,成为了一种重要的学习方式。针对传统作业中存在的问题,可以把传统的家庭作业和移动互联网技术结合起来,老师、学生和家长都能利用App一键检查作业,同时把错误问题上传到服务器,既节省了老师批改作业的时间,也记录下了学生的易错点,可以为后续教学的展开提供很大的帮助。
\par
因此,开发一款能够提供检查作业功能的App,能够为解决当前小学数学作业中存在的问题提供协助,所以本文针对算术运算检查、班级管理、错误记录反馈、统计分析、趣味练习题等需求结合深度学习和移动App技术,设计并实现了一款数学教学微信小程序。


\section{课题研究现状}
\subsection{识别字符算法的研究现状}
算式识别模块是基于图像处理、模式识别等技术的综合系统,一般流程是首先进行图像收集,然后对图像进行预处理,包括图像二值化、形态学处理、字符分割、归一化等操作,目的是突出字符区域,为后续的字符识别做准备;一组算式中多个字符的图像根据字符间隙, 进行字符分割, 分割成单个字符进行识别。由于相邻数字的重叠和连接,字符识别可能会比较困难,为了解决这个问题,可以使用基于定向滑动窗口的手写连接数字的分段识别法,这种方法允许通过同时找到相邻数字和切割路径之间的互连点来根据连接配置分离相邻数字。最终可以把提取的字符特征输入深度学习模型中进行字符识别,判断算式的结果是否正确。
\par
字符识别可以使用多种机器学习算法实现,各有优缺点。支持向量机在小样本学习中可以得到很高的准确率; BP神经网络来通过不断调整神经各层间的权值和阈值来使实际样本的期望值和实际输出值的方差的平均值最小来实现数字识别,具有自学习能力,但学习速度慢,容易陷入局部极值;CNN的网络拓扑结构能和图像很好的吻合,特征分类效果好;KNN重新训练代价低但计算量大,样本不平衡时预测偏差较大;决策树易于理解和解释、运行速度快,但缺失数据处理困难,容易出现过拟合;朴素贝叶斯可以处理多分类问题,对小规模数据表现好,但对数据形式敏感分类决策存在错误率。
\par
从已经发表的文章中的工作来看,现在的研究点多集中于单个字符的识别或单个算式的识别,对一张图片上多个算式的识别,公开的成果不多,因此研究并实现这一算法,具有一定的现实意义。

\subsection{移动App研究现状}
移动App按照开发模式可以大体分为Native App、Web App和Hybrid App三种。Native App是使用原生语言编写第三方应用程序,现有IOS和Android两种。它运行在移动终端的操纵系统中,可以直接调用操作系统的API,性能较好,可以提供最优的用户体验。Native App虽然性能卓越,但开发成本也较高,维护更新复杂,而且不支持跨平台使用,可移植性差,代码重用率低,开发门槛较高,不适合个人开发者进行开发。
\par
Web App是运行在浏览器上的应用,通过把页面部署到服务器中,用户使用浏览器访问,无需下载安装,解决了Native App无法跨平台使用的问题。但Web App的缺点也十分明显,对网络的依赖很大,访问页面需要到服务器加载资源,速度较慢;Web App在浏览器中运行,受限于HTML5的技术特性,几乎无法使用设备的本地资源,很多功能无法实现,用户体验较差。
\par
Hybrid App是Native App和Web App两者混合的产物,实际就是Native的框架加上Web的内容。它既能利用移动设备的API带来良好的用户体验,也具有Web App跨平台、高效开发、快速发布的优点,Web内容可以做到一次发布所有平台生效,使用网页语言编码开发难度较小、效率较高。微信小程序作为一种特殊的Hybrid App不受手机操作系统的限制,可以调用手机系统的功能,在性能上接近Native App。而且国内微信用户数目众多,加上小程序应用场景广泛,培养了很多用户群体。综合考虑以上因素,微信小程序既保证了平台的通用性、易于开发,又有广泛的用户群体,最适合本文用来实现辅助教学App。

\section{论文组织结构}
本论文主要研究

本文的组织结构如下:

第一章:绪论。在绪论章节中,本文首先阐述了本课题的研究背景,

第二章:相关知识和工作介绍。

第三章:

第四章:

第五章:

第六章:

第七章:

第八章: