\chapter{系统测试}
\section{字符识别模型准确率的测试}
字符识别模型在测试集上的准确率可以达到99.776\%,但经过预处理导入模型识别的图片和训练集中的图片存在一定的偏差,识别的准确率可能会因此降低。所以需要使用含多个字符的图片去测试,本文使用含有50个手写字符的图片测试识别模型。测试结果见表7.1。
\par
\begin{table}[h!]
\begin{center}
\caption{模型准确率测试}
\begin{tabular}{|c|c|c|c|}
 \hline
 字符名 & 识别准确率 & 测试样本数 & 错误识别结果\\
 \hline
0 & 100\% & 150 & none \\
 \hline
1 & 98\% & 100 & ( and ) \\
 \hline
2 & 99\% & 100 & 4 \\
 \hline
 3 & 99\% & 100 & ( \\
  \hline
 4 & 100\% & 100 & none \\
  \hline
 5 & 98\% & 100 & ( and ) \\
  \hline
 6 & 97\% & 100 & ( and 5 \\
  \hline
 7 & 91\% & 100 & ), + and 1 \\
  \hline
 8 & 100\% & 100 & none \\
  \hline
 9 & 100\% & 100 & none \\
  \hline
 + & 100\% & 100 & none \\
  \hline
 - & 100\% & 100 & none \\
  \hline
 * & 100\% & 100 & none \\
  \hline
$\backslash$ & 97\% & 100 & = and 2 \\
  \hline
 = & 97.2\% & 110 & 2 and 5 \\
  \hline
 ( & 100\% & 110 & none \\
  \hline
 ) & 98\% & 100 & 7 \\
  \hline
\end{tabular}
\end{center}
\end{table}
\par
由上表可以看出大部分字符的识别准确率可以达到97\%以上,形状上与其他字符差别很大的字符可以达到很高的准确率,数字1、6和7因为和左右括号在形状上相似,识别准确率相对于其他字符较低,通过在数据集中增加图片,来增强数据集可以提高识别准确率。同时,图片预处理的结果对模型识别的准确率有很大的影响。因为图片拍摄角度和对焦的原因,图片的一部分可能会比较模糊,在预处理切分成单个字符的时候可能把单个字符分隔开,或者因为两行字符在同一水平线上,导致两个字符被分割在一起,进而识别错误。


\section{识别算式准确率的测试}

\section{App功能测试}

\section{服务端性能测试}



